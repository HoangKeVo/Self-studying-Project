\documentclass[12pt, a4paper]{article}
\usepackage{amsfonts, amsmath, amssymb, amsthm}
\usepackage{enumitem}
\usepackage{mathtools}
\usepackage{fullpage}
\theoremstyle{plain}
\newtheorem{innercustomgeneric}{\customgenericname}
\providecommand{\customgenericname}{}
\newcommand{\newcustomtheorem}[2]{%
\newenvironment{#1}[1]{
\renewcommand\customgenericname{#2}%
\renewcommand\theinnercustomgeneric{##1}%
\innercustomgeneric
}
{\endinnercustomgeneric}
}
\newcustomtheorem{exercise}{Exercise}
\newcustomtheorem{lemma}{Lemma}

\newcommand{\N}{\mathbb{N}}
\newcommand{\Q}{\mathbb{Q}}
\newcommand{\R}{\mathbb{R}}
\newcommand{\Z}{\mathbb{Z}}
\newcommand{\C}{\mathbb{C}}

\begin{document}
\begin{exercise}{21}
Show that any ternary decimal of the form $0.a_1a_2\cdots a_n11$ (base 3) is not an element of $\Delta$.
\end{exercise}
	\begin{proof}
	Because $0.a_1a_2\cdots a_n1<0.a_1a_2\cdots a_n11<0.a_1a_2\cdots a_n12$ and $(0.a_1a_2\cdots a_n1,0.a_1a_2\cdots a_n12)$ will be chopped off, thus $0.a_1a_2\cdots a_n11$ is not an element of $\Delta$.
	\end{proof}
	
\begin{exercise}{22}
Show that $\Delta$ contains no (nonempty) open intervals.
\end{exercise}
	\begin{proof}
	We have the total length of $[0,1]\setminus\Delta=1$. Thus $\Delta$ cannot contain any open interval.
	\end{proof}

\begin{exercise}{23}
Show that the endpoints of $\Delta$ other that 0 and 1 can be written as $0.a_1a_2\cdots a_{n+1}$ (base 3), where each $a_k$ is 0 or 2, except $a_{n+1}$, which is either 1 or 2. That is, the discarded "middle third" intervals are of the form $(0.a_1a_2\cdots a_n1,\,0.a_1a_2\cdots a_n2)$, where both entries are points of $\Delta$ written in base 3.
\end{exercise}
	\begin{proof}
	We will prove the part "that is" using mathematical induction. For $n=1$, the discarded interval is $(0.1,0.2)$, and for $n=1$, the discarded intervals are $(0.01,0.02)$ and $(0.21,0.22)$. Thus the hypothesis holds. Assume that the hypothesis holds for $n=i$, that is, the discarded interval has the form $(0.a_1a_2\cdots  a_i1,\, 0.a_1a_2\cdots a_i2)$ where both entries are points of $\Delta$, we will prove that it is also hold for $n=i+1$. 
	
	Notice that the discarded interval in the $i+1$th step cannot be $(0.a_1a_2\cdots  a_{i+1}0,\, 0.a_1a_2\cdots a_{i+1}1)$ because it is the first third of the remaining interval $(0.a_1a_2\cdots  a_{i+1},\, 0.a_1a_2\cdots a_{i+1}1)$ in the $i$th step. It cannot be $(0.a_1a_2\cdots  a_{i+1}2,\, (0.a_1a_2\cdots a_{i+1}+\frac{1}{3^{i+2}}))$ because it is the last third of the remaining interval $(0.a_1a_2\cdots  a_{i+1},\, (0.a_1a_2\cdots a_{i+1}+\frac{1}{3^{i+2}}))$. Thus it can only have the form $(0.a_1a_2\cdots  a_{i+1}1,\, 0.a_1a_2\cdots a_{i+1}2)$.
	\end{proof}

\begin{exercise}{24}
Show that $\Delta$ is perfect; that is, every point is $\Delta$ is the limit of a sequence of distinct points from $\Delta$. In fact, show that every point in $\Delta$ is the limit of a sequence of distinct endpoints.
\end{exercise}
	\begin{proof}
	We will split a point in $\Delta$ into two cases. Case 1, it contains finitely many digits (base 3). That means, it can be represented as $0.a_1a_2\cdots a_n$ be a point in $\Delta$. If $a_n=1$, defined a sequence $x_i=0.a_1a_2\cdots a_n-\frac{1}{3^n}+\frac{2}{3^{n+1}}+\frac{2}{3^{n+2}}+\cdots +\frac{2}{3^{n+i}}$. Clearly, $x_i\rightarrow 0.a_1a_2\cdots a_n$ and for any $i$, $x_i$ contains only 0 and 2. Thus $x_i$ is an end point.
	
	If $a_n=2$, we defined $x_i=0.a_1a_2\cdots a_n+\frac{1}{3^n} -\frac{2}{3^{n+1}} -\frac{2}{3^{n+2}} -\cdots -\frac{2}{3^{n+i}}$. Clearly, $x_i\rightarrow 0.a_1a_2\cdots a_n$ and $x_i$ contains only 0's and 2's except for the last digit, which is 1. Thus $x_i\in \Delta$.
	
	If $a_n=0$, then that number is either 1 or 0. If $0.a_1a_2\cdots a_n=0$, then let $(x_i)=\frac{1}{3^i}$ and if $0.a_1a_2\cdots a_n=1$ then $(x_i)=1-\frac{1}{3^i}$. We can easily check that $x_i\in \Delta$ and $x_i\rightarrow 0.a_1a_2\cdots a_n$.
	
	If that point contains infinitely many digits: $0.a_1a_2\cdots$, then $a_i$ can be either 0 or 2 and there are infinitely many 2's. Let $(x_i)=0.a_1a_2\cdots a_{t_i}$ such that $x_i$ contains $i$ 2's. Thus $x_i\in \Delta$ and $x_i \rightarrow 0.a_1a_2\cdots$.
	
	Thus there is always exist a sequence of points in $\Delta$ that converges to a point in $\Delta$.
	\end{proof}
	
\pagebreak

\begin{exercise}{26}
Let $f:\Delta\rightarrow [0,1]$ be the Cantor function and let $x,y\in \Delta$ with $x<y$. Show that $f(x)\leq f(y)$. If $f(x)=f(y)$, show that $x$ has two distinct binary decimal expansions. Finally, show that $f(x)=f(y)$ if and only if $x$ and $y$ are consecutive endpoints of the form $x=0.a_1a_2\cdots a_n1$ and $y=0.a_1a_2\cdots a_n2$ (base 3).
\end{exercise}
	\begin{proof}
	Any endpoints $t$ of $\Delta$ has the form 
	\[\sum_{n=1}^{\infty}{\frac{2b_{t,n}}{3^n}}
	\]
	for $b_n=0,\,1$. Thus if $x<y$, then there exists an integer $k$ such that $b_{x,1}=b_{y,1},b_{x,2}=b_{y,2},\cdots ,b_{x,k-1}=b_{y,k-1}$ and $b_{x,k}<b_{y,k}$. Thus \[
	\sum_{n=1}^{\infty}{\frac{2b_{x,n}}{2^n}}\leq\sum_{n=1}^{\infty}{\frac{2b_{y,n}}{2^n}}
	\]
	or $f(x)\leq f(y)$. If $f(x)=f(y)$ then $x$ and $y$ must be of the forms $0.b_{x,1}\cdots b_{x,k-1}0111\cdots$ and $0.b_{y,1}\cdots b_{y,k-1}1$ (base 2). Thus $x=0.b_{x,1}\cdots b_{x,k-1}1\cdots$ and $y=0.b_{y,1}\cdots b_{y,k-1}2$ (base 3).
	\end{proof}

\begin{exercise}{27}
Fix $n\geq 1$, and let $I_{n,k},k=1,\cdots ,2^{n-1}$ be the component subintervals of the $n$th level Cantor set $I_n$. If $x,y\in\Delta$ with $|x-y|<3^{-n}$, show that $x$ and $y$ are in the same component $I_{n,k}$. For this same pair of points show that $|f(x)-f(y)|\leq 2^{-n}$.
\end{exercise}
	\begin{proof}
	To show that $x$ and $y$ are in the same component $I_{n,k}$, we will prove that after the $n$th step, any removed interval has the length of $3^{-n}$ or above. Indeed, by exercise 23, the middle third intervals have the form $(0.a_1a_2\cdots a_{t}1,0.a_1a_2\cdots a_{t}2)$ where $t=0,\, 1,\cdots ,\,n-1$. Thus all the intervals are longer or equal to $3^{-n}$. Thus if $|x-y|<3^{-n}$, $x,y$ must be in the same component $I_{n,k}$. 
	
	If $x,y$ are in the same component $I_{n,k}$, then either $I_{n,k}=(0.a_1a_2\cdots a_{n-1}0,0.a_1a_2\cdots a_{n-1}1)$ or $I_{n,k}=(0.a_1a_2\cdots a_{n-1}2,0.a_1a_2\cdots a_{n-1}2+3^{-n})$. In the first case, we have
	\[
	|f(x)-f(y)|\leq |0.a_1a_2\cdots a_{n-1}0-0.a_1a_2\cdots a_{n-1}1| \text{ (base 2)}=2^{-n}.
	\]
	Similar with the second case, we have $|f(x)-f(y)|\leq 2^{-n}$.
	\end{proof}
	
\begin{exercise}{28}
Let $f:\Delta\rightarrow [0,1]$ be then Cantor function (as originally defined). Check that $f(x)=\sup \{f(y):y\in\Delta ,y\leq x\}$ for any $x\in\Delta$.
\end{exercise}
	\begin{proof}
	In exercise 26, we already 	proved that if $x,y\in\Delta$ and $x<y$ then $f(x)\leq f(y)$. Now, for any $f(v)\in \{f(y):y\in\Delta ,y\leq x\}$, if $v<x$ then $f(v)\leq f(x)$. if $v=x$ then $f(v)=f(x)$. Thus $f(x)$ is an upper bound for $\{f(y):y\in\Delta ,y\leq x\}$. Moreover, $f(x)\in\{f(y):y\in\Delta ,y\leq x\}$, thus $\sup\{f(y):y\in\Delta ,y\leq x\}=f(x)$ for any $x\in \Delta$.
	\end{proof}

\begin{exercise}{29}
Prove that the extended Cantor function $f:[0,1]\rightarrow [0,1]$ is increasing.
\end{exercise}
	\begin{proof}
	For any $x,y\in [0,1]$ and $x<y$, we have $\{f(t):t\in\Delta ,t\leq x\}\subset\{f(t):t\in\Delta ,t\leq y\}$. Thus $f(x)=\sup\{f(t):t\in\Delta ,t\leq x\}\leq \sup\{f(t):t\in\Delta ,t\leq y\}=f(y)$. Thus $f$ is increasing.
	\end{proof}
	
\begin{exercise}{30}
Check that the construction of the generalized Cantor set with parameter $\alpha$ leads to a set of measure $1-\alpha$; that is, check that the discarded intervals now have total length $\alpha$.
\end{exercise}
	\begin{proof}
	At the $n$th stage, we discard $2^{n-1}$ intervals, each has length $\alpha 3^{-n}$, thus the total length of the discarded interval is
	\begin{align*}
	\sum_{n=1}^{\infty}{\alpha 2^{n-1}3^{-n}}&=\alpha\sum_{n=1}^{\infty}{2^{n-1}3^{-n}}\\
	&=\frac{1}{2}\alpha\sum_{n=1}^{\infty}{\alpha 2^{n}3^{-n}}\\
	&=\frac{1}{2}\alpha\left(\frac{\frac{2}{3}^{\infty}-1}{\frac{2}{3}-1}-1\right)\\
	&=	\alpha.
	\end{align*}
	Thus the discarded interval has length $\alpha$.
	\end{proof}
	
\begin{exercise}{32}
Deduce from Theorem 2.17 that a monotone function $f:\R\rightarrow\R$ has points of continuity in every open interval.
\end{exercise}
	\begin{proof}
	First, remind that theorem 2.17 states as follow, if $f : (a, b)\rightarrow\R$ is monotone, then $f$ has at most countably many points of discontinuity in $(a, b)$, all of which are jump discontinuities.
	
	Back to the problem, for a monotone function $f:\R\rightarrow\R$, assume that there exists an open interval $(a,b)$ such that $f$ discontinuous at every point in $(a,b)$. Then the set of discontinuity points consists of $\{x:a<x<b\}$. And since this set is uncountable, the set of discontinuity points of $f$ is also uncountable, which is contradict to theorem 2.17. Thus $f$ must have points of continuity in every open interval.
	\end{proof}
	
\begin{exercise}{34}
	Let $D =\{x_1, x_2,\cdots\}$ , and let $\epsilon_n > 0$ with $\sum_{n=1}^\infty{\epsilon_n}<\infty$. Define $f(x) = \sum_{x_n\leq x}{\epsilon_n}$. Check the following: 
	\begin{enumerate}[label=(\roman*)]
	\item  $f$ is discontinuous at the points of $D$.
		\begin{proof}
		For any $x_i$, we have $f(x_i-)=\sum_{x_n<x_i}\epsilon_n$ and $f(x_i)=\sum_{x_n\leq x_i}\epsilon_n=\sum_{x_n<x_i}\epsilon_n+\epsilon_i>f(x_i-)$. Thus $f$ is discontinuous at the points of $D$.
		\end{proof}
	\item $f$ is right-continuous everywhere.
		\begin{proof}
		For any $t$, if $t\notin D$ then $f$ continuous at $t$, thus $f$ is right-continuous at $t$. If $t\in D$, assume that $t=x_i$. We will prove that $f(x_i)$ is right-continuous using the definition.
		
		By the definition, $f$ is right-continuous at $x_i$ if and only if for any $\epsilon >0$, there exists $\delta >0$ such that, if $x_i<y<x_i+\delta$, then $|f(y)-f(x)|<\epsilon$, or
		\begin{align*}
		&|\sum_{x_n\leq y}\epsilon_n-\sum_{x_n\leq x_i}\epsilon_n|<\epsilon\\
		\Leftrightarrow &\sum_{x_i<x_n\leq y}\epsilon_n<\epsilon .		
		\end{align*}
	We know that 
	\[
	\sum_{x_i<x_n\leq y}\epsilon_n\leq \sum_{x_i<x_n\leq x_i+\delta}\epsilon_n ,
	\]
	thus it's sufficient to prove that 
	\[
	\sum_{x_i<x_n\leq x_i+\delta}\epsilon_n <\epsilon
	\]
	for some $\delta >0$. Notice that $\sum_{n=1}^{\infty}{\epsilon_n}<\infty$, thus 
	\[
	\lim_{N\rightarrow\infty}{\sum_{n=N}^{\infty}{\epsilon_n}}=0,
	\]
	which means, by the definition, there exists $\delta'\in\N$ such that 
	\[
	\sum_{n=\delta'}^{\infty}{\epsilon_n}<\epsilon.
	\]
	Now, let $\delta =\min\{x_t|t\leq\delta' \text{ and }x_t>x_i\}-x_i$, then for any $x\in\{x_1,x_2,\cdots,x_{\delta'}\}$, either $x\leq x_i$ or $\delta < x-x_i$, which means $x_i+\delta<x$. Thus $(x_i,x_i+\delta)\cap\{x_1,x_2,\cdots,x_{\delta'}\}=\varnothing$. Thus 
	\[
	\sum_{x_i<x_n\leq x_i+\delta}\epsilon_n < \sum_{n=\delta'}^{\infty}\epsilon_n <\epsilon.
	\]
	Since such $\delta$ exists, $f$ is always right-continuous.
		\end{proof}
	\item $f$ is continuous at each point $x\in \R\setminus D$.
How might this construction be modified so as to yield a strictly increasing function with these same properties?
		\begin{proof}
		Since $f$ is already right-continuous everywhere, it's sufficient to prove that $f(x)$ is left-continuous for any $x\in\R\setminus D$. Indeed, we have 
		\[
		f(x)=\sum_{x_n\leq x}{\epsilon_n}=\sum_{x_n< x}{\epsilon_n}
		\]
		since $x_n\neq x$ for all $n$. Moreover, 
		\[
		f(x-)=\lim_{t\rightarrow x^-}\sum_{x_n\leq t}{\epsilon_n}=\sum_{x_n< x}{\epsilon_n}.
		\]
		Clearly, $f(x)=f(x-)$, we get $f(x)$ is also left-continuous. Thus $f(x)$ is continuous for all $x\in\R\setminus D$.
		
		Let $g(x)=x+f(x)$, since $g'(x)=1$, $g$ is strictly increase, and $g$ has the same continuous properties as $f$.
		\end{proof}
	\end{enumerate} 
\end{exercise}

\begin{exercise}{35}
Let $f:[a,b]\rightarrow\R$ be increasing, and let $(x_n)$ be an enumeration of the discontinuities of $f$. For each $n$, let $a_n=f(x_n)-f(x_n-)$ and $b_n=f(x_n+)-f(x_n)$ be the left and right "jumps" in the graph of $f$, where $a_n=0$ if $x_n=a$ and $b_n=0$ if $x_n=b$. Show that $\sum_{n=1}^\infty a_n\leq f(b)-f(a)$ and $\sum_{n=1}^{\infty}b_n\leq f(b)-f(a)$. 
\end{exercise}
	\begin{proof}
	First, notice that
	\begin{align*}
	\sum_{n=1}^{\infty}{a_n+b_n}&=f(x_1)-f(x_1-)+f(x_1+)-f(x_1)+\cdots\\
	&=f(x_1+)-f(x_1-)+f(x_2+)-f(x_2-)+\cdots.
	\end{align*}
	Now, we will prove that for the set $X=\{x_n|x_n \text{ is a point of discontinuity of $f$ }\}$, we always have 
	\[
	\sum_{n=1}^{|X|}{a_n+b_n}\leq f(\sup(X)+)-f(\inf(X)-).
	\]
	Indeed, if $|N|=1$, then the result becomes $f(x_1+)-f(x_1-)\leq f(x_1+)-f(x_1-)$, which is trivial. Assume that the result holds for $|X|=k$, we will prove that for $|X|=k+1$, this result also holds. Indeed, for $X=\{x_1,x_2,\cdots ,x_{k+1}\}$, without loss of generality, assume that $\inf(X)=x_1$. And since $f$ is increasing, we have $f(x_1+)\leq f(\inf(X\setminus\{x_1\})$. Thus
	\begin{align*}
	\sum_{n=1}^{k+1}{a_n+b_n}&=f(x_1+)-f(x_1-)+\sum_{n=2}^{k+1}{f(x_n+)-f(x_n-)}\\
	&\leq f(x_1+)-f(x_1-)+f(\sup(X\setminus\{x_1\})-f(\inf(X\setminus\{x_1\})\\
	&\leq f(\sup(X\setminus\{x_1\})-f(x_1-)\\
	&= f(\sup(X)+)-f(\inf(X)-).
	\end{align*}
	For the case $|X|=\infty$, let $X_k=\{x_1,x_2,\cdots ,x_k\}$, we have
	\[
	\sum_{n=1}^{k}{a_n+b_n}\leq f(\sup(X_k)+)-f(\inf(X_k)-).
	\]
	Take the limit both sides, we get
	\[
	\lim_{k\rightarrow\infty}\sum_{n=1}^{k}{a_n+b_n}\leq \lim_{k\rightarrow\infty} f(\sup(X_k)+)-\lim_{k\rightarrow\infty}f(\inf(X_k)-).
	\]
	Notice that since $\sup(X_k)$ is increasing and $\inf(X_k)$ is decreasing as $k\rightarrow\infty$, the sequence $f(\sup(X_k)+)-f(\inf(X_k)-)$ is increasing and be bounded by $f(\sup[a,b]+)-f(\inf[a,b]-)=f(a)-f(b)$. Thus $\lim_{k\rightarrow\infty} f(\sup(X_k)+)-f(\inf(X_k)-)$ is converge. Thus the sequence $\sum_{n=1}^{k}{a_n+b_n}$ is bounded by $f(a)-f(b)$ too. Moreover, because $a_n,b_n\geq 0$, $\sum_{n=1}^{k}{a_n+b_n}$ is increasing, thus it is also converge. Thus
	\[
	\sum_{n=1}^{\infty}{a_n+b_n}\leq f(\sup(X)+)-f(\inf(X)-).
	\]
	Now, since $f$ is increasing and $\sup(X),\inf(X)\in [a,b]$, we always have $f(\sup(X)+)-f(\inf(X)-)\leq f(b)-f(a)$. Thus
	\[
	\sum_{n=1}^{|X|}{a_n+b_n}\leq f(b)-f(a),
	\]
	for $X$ finite and
	\[
	\sum_{n=1}^{\infty}{a_n+b_n}\leq f(b)-f(a),
	\]
	for $X$ infinite, which is a stronger result, compared to $\sum_{n=1}^\infty a_n\leq f(b)-f(a)$ and $\sum_{n=1}^{\infty}b_n\leq f(b)-f(a)$. 
	\end{proof}
	
\begin{exercise}{36}
In the notation of Exercise 35, define $h(x)=\sum_{x_n\leq x}a_n+\sum_{x_n<x}{b_n}$. Show that $h$ is increasing and that $g = f - h$ is both continuous and increasing. Thus, each increasing function $f$ can be written as the sum of a continuous increasing function
$g$ and a "pure jump" function $h$.
\end{exercise}
	\begin{proof}
	Since both $a_n$ and $b_n$ is non-negative,
	\[
	h(x)=\sum_{x_n\leq x}a_n+\sum_{x_n<x}{b_n}
	\]
	is increasing. For any $x>y$, we have 
	\[
	g(x)-g(y)=(f(x)-f(y))-(h(x)-h(y)).
	\]
	But by exercise 35, we know that
	\[
	f(x)-f(y)\geq \sum_{y\leq x_n\leq x}{a_n+b_n},
	\]
	thus $g(x)-g(y)>0$, which means $g$ is increasing. Now, if $x\notin X$, then $f(x)$ and $h(x)$ is continuous. Thus $g(x)=f(x)-h(x)$ is also continuous. If $x\in X$, let $x=x_i$, we will prove that $g(x-)=g(x)=g(x+)$. Indeed,
	\begin{align*}
	g(x_i)-g(x_i-)&=(f(x_i)-f(x_i-))-(h(x_i)-h(x_i-))\\
	&=a_i-(\sum_{x_n\leq x_i}a_n+\sum_{x_n< x_i}b_n-\sum_{x_n< x_i}a_n-\sum_{x_n< x_i}b_n)\\
	&=a_i-a_i\\
	&=0.
	\end{align*}
	Thus $g(x_i)=g(x_i-)$. Moreover, we have
	\begin{align*}
	g(x_i+)-g(x_i)&=(f(x_i+)-f(x_i))-(h(x_i+)-h(x_i))\\
	&=b_i-(\sum_{x_n\leq x_i}a_n+\sum_{x_n\leq x_i}b_n-\sum_{x_n\leq x_i}a_n-\sum_{x_n< x_i}b_n)\\
	&=b_i-b_i\\
	&=0,
	\end{align*}
	which means $g(x_i+)=g(x)$. Thus $g$ is continuous.
	\end{proof}
\end{document}
